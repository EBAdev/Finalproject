\documentclass[a4paper,oneside,article,english]{memoir}


%lav margins mindre (bedre til normale afleveringer)
\setlrmarginsandblock{2.5cm}{*}{1} \setulmarginsandblock{3cm}{3cm}{*} \checkandfixthelayout


\usepackage[utf8]{inputenc}  % Korrekt håndtering af æ, ø og å
\usepackage[T1]{fontenc}  % Korrekt håndtering af æ, ø og å
\usepackage{microtype}  % Typografi! Giver bl.a. pænere orddeling
\usepackage[english]{babel}  % Danske betegnelser og orddeling

\usepackage{amsmath, amssymb,amsfonts, bm, mathtools}  % Giver adgang til matematikting
\usepackage[Gray,squaren,thinqspace,thinspace]{SIunits} % Gør det muligt at angive enheder

\usepackage{graphicx}  % Gør det muligt at indsætte billeder
\usepackage{url} % Bruges til indsættelse af url direkte
\usepackage{float} % gør det muligt at tvinge en figurs placering med [H]
\usepackage[small,bf,hang]{caption}  %Gør caption bold
\usepackage{subcaption} % Gør det muligt at lave delfigurer
\usepackage{lipsum}
\usepackage{dsfont} % giver adgang til\mathds{#} for at lave bb numre
\usepackage{parskip}

% TikZ
\usepackage{tikz} % Bruges til at lave flotte figurer
\usetikzlibrary{calc} % bruges til at lave nem placering i tikz
\usepackage{pgfplots} % bruges til plots i TikZ
\pgfplotsset{compat=1.18}

\tikzset{
    % Two node styles for game trees: solid and hollow
    solid node/.style={circle,draw,inner sep=1.5,fill=black},
    hollow node/.style={circle,draw,inner sep=1.5}
}

%%%%%%%%%%%%%%%%%%%%
% Begin enviorment setup			 %
%%%%%%%%%%%%%%%%%%%%
\usepackage{amsthm}
\usepackage{thmtools}
\usepackage[many]{tcolorbox}
\makeatletter

\def\renewtheorem#1{%
  \expandafter\let\csname#1\endcsname\relax
  \expandafter\let\csname c@#1\endcsname\relax
  \gdef\renewtheorem@envname{#1}
  \renewtheorem@secpar
}
\def\renewtheorem@secpar{\@ifnextchar[{\renewtheorem@numberedlike}{\renewtheorem@nonumberedlike}}
\def\renewtheorem@numberedlike[#1]#2{\newtheorem{\renewtheorem@envname}[#1]{#2}}
\def\renewtheorem@nonumberedlike#1{
  \def\renewtheorem@caption{#1}
  \edef\renewtheorem@nowithin{\noexpand\newtheorem{\renewtheorem@envname}{\renewtheorem@caption}}
  \renewtheorem@thirdpar
}
\def\renewtheorem@thirdpar{\@ifnextchar[{\renewtheorem@within}{\renewtheorem@nowithin}}
\def\renewtheorem@within[#1]{\renewtheorem@nowithin[#1]}

\makeatother

\usepackage[framemethod=TikZ]{mdframed}
\mdfsetup{skipabove=1em,skipbelow=0em, innertopmargin=12pt, innerbottommargin=8pt}

\tcbuselibrary{skins}

% Color definitions

\definecolor{proofcolor}{RGB}{0,0,0}

% Dark orange and Dark Red rgb
\definecolor{theorembordercolor}{RGB}{151, 63, 5}
\definecolor{theorembackgroundcolor}{RGB}{248, 241, 234}

\definecolor{examplebordercolor}{RGB}{0, 110, 184}
\definecolor{examplebackgroundcolor}{RGB}{240, 244, 250}

\definecolor{definitionbordercolor}{RGB}{0, 150, 85}
\definecolor{definitionbackgroundcolor}{RGB}{239, 247, 243}

\definecolor{propertybordercolor}{RGB}{128, 0, 128}
\definecolor{propertybackgroundcolor}{RGB}{255, 240, 255}

\definecolor{formulabordercolor}{RGB}{0, 0, 0}
\definecolor{formulabackgroundcolor}{RGB}{230, 229, 245}

%%% THEOREM STYLE SETUP %%%
\newtheoremstyle{theorem}
{0pt}{0pt}{\normalfont}{0pt}
{}{\;}{0.25em}
{{\sffamily\bfseries\color{theorembordercolor}\thmname{#1}~\thmnumber{\textup{#2}}.}
  \thmnote{\normalfont\color{black}~(#3)}}

\newtheoremstyle{definition}
{0pt}{0pt}{\normalfont}{0pt}
{}{\;}{0.25em}
{{\sffamily\bfseries\color{definitionbordercolor}\thmname{#1}~\thmnumber{\textup{#2}}.}
  \thmnote{\normalfont\color{black}~(#3)}}

\newtheoremstyle{example}
{0pt}{0pt}{\normalfont}{0pt}
{}{\;}{0.25em}
{{\sffamily\bfseries\color{examplebordercolor}\thmname{#1}~\thmnumber{\textup{#2}}.}
  \thmnote{\normalfont\color{black}~(#3)}}

\newtheoremstyle{property}
{0pt}{0pt}{\normalfont}{0pt}
{}{\;}{0.25em}
{{\sffamily\bfseries\color{propertybordercolor}\thmname{#1}~\thmnumber{\textup{#2}}.}
  \thmnote{\normalfont\color{black}~(#3)}}

\newtheoremstyle{formula}
{0pt}{0pt}{\normalfont}{0pt}
{}{\;}{0.25em}
{{\sffamily\bfseries\color{formulabordercolor}\thmname{#1}~\thmnumber{\textup{#2}}.}
  \thmnote{\normalfont\color{black}~(#3)}}

%%%%%%%%%%%%%%%%%%%%%%%%
% Theorem Environments 				%
%%%%%%%%%%%%%%%%%%%%%%%%

\theoremstyle{theorem}
\newtheorem{theorem}{Theorem}
\newtheorem{postulate}{Postulate}
\newtheorem{conjecture}{Conjecture}
\newtheorem{corollary}{Corollary}
\newtheorem{lemma}{Lemma}
\newtheorem{conclusion}{Conclusion}

\tcolorboxenvironment{theorem}{
  enhanced jigsaw, pad at break*=1mm, breakable,
  left=4mm, right=4mm, top=1mm, bottom=1mm,
  colback=theorembackgroundcolor, boxrule=0pt, frame hidden,
  borderline west={0.5mm}{0mm}{theorembordercolor}, arc=.5mm
}
\tcolorboxenvironment{postulate}{
  enhanced jigsaw, pad at break*=1mm, breakable,
  left=4mm, right=4mm, top=1mm, bottom=1mm,
  colback=theorembackgroundcolor, boxrule=0pt, frame hidden,
  borderline west={0.5mm}{0mm}{theorembordercolor}, arc=.5mm
}
\tcolorboxenvironment{conjecture}{
  enhanced jigsaw, pad at break*=1mm, breakable,
  left=4mm, right=4mm, top=1mm, bottom=1mm,
  colback=theorembackgroundcolor, boxrule=0pt, frame hidden,
  borderline west={0.5mm}{0mm}{theorembordercolor}, arc=.5mm
}
\tcolorboxenvironment{corollary}{
  enhanced jigsaw, pad at break*=1mm, breakable,
  left=4mm, right=4mm, top=1mm, bottom=1mm,
  colback=theorembackgroundcolor, boxrule=0pt, frame hidden,
  borderline west={0.5mm}{0mm}{theorembordercolor}, arc=.5mm
}
\tcolorboxenvironment{lemma}{
  enhanced jigsaw, pad at break*=1mm, breakable,
  left=4mm, right=4mm, top=1mm, bottom=1mm,
  colback=theorembackgroundcolor, boxrule=0pt, frame hidden,
  borderline west={0.5mm}{0mm}{theorembordercolor}, arc=.5mm
}
\tcolorboxenvironment{conclusion}{
  enhanced jigsaw, pad at break*=1mm, breakable,
  left=4mm, right=4mm, top=1mm, bottom=1mm,
  colback=theorembackgroundcolor, boxrule=0pt, frame hidden,
  borderline west={0.5mm}{0mm}{theorembordercolor}, arc=.5mm
}

%%%%%%%%%%%%%%%%%%%%%%%%%%%
% Definition Environments %
%%%%%%%%%%%%%%%%%%%%%%%%%%%

\theoremstyle{definition}
\newtheorem{definition}{Definition}
\newtheorem{review}{Review}

\tcolorboxenvironment{definition}{
  enhanced jigsaw, pad at break*=1mm, breakable,
  left=4mm, right=4mm, top=1mm, bottom=1mm,
  colback=definitionbackgroundcolor, boxrule=0pt, frame hidden,
  borderline west={0.5mm}{0mm}{definitionbordercolor}, arc=.5mm
}
\tcolorboxenvironment{review}{
  enhanced jigsaw, pad at break*=1mm, breakable,
  left=4mm, right=4mm, top=1mm, bottom=1mm,
  colback=definitionbackgroundcolor, boxrule=0pt, frame hidden,
  borderline west={0.5mm}{0mm}{definitionbordercolor}, arc=.5mm
}


%%%%%%%%%%%%%%%%%%%%%%%%
% Example Environments 			%
%%%%%%%%%%%%%%%%%%%%%%%%

\theoremstyle{example}
\newtheorem{example}{Example}
\newtheorem{remark}{Remark}
\newtheorem{note}{Note}
\newtheorem{claim}{Claim}
\newtheorem{fact}{Fact}

\tcolorboxenvironment{example}{
  enhanced jigsaw, pad at break*=1mm, breakable,
  left=4mm, right=4mm, top=1mm, bottom=1mm,
  colback=examplebackgroundcolor, boxrule=0pt, frame hidden,
  borderline west={0.5mm}{0mm}{examplebordercolor}, arc=.5mm
}
\tcolorboxenvironment{remark}{
  enhanced jigsaw, pad at break*=1mm, breakable,
  left=4mm, right=4mm, top=1mm, bottom=1mm,
  colback=white, boxrule=0pt, frame hidden,
  borderline west={0.5mm}{0mm}{examplebordercolor}, arc=.5mm
}
\tcolorboxenvironment{note}{
  enhanced jigsaw, pad at break*=1mm, breakable,
  left=4mm, right=4mm, top=1mm, bottom=1mm,
  colback=white, boxrule=0pt, frame hidden,
  borderline west={0.5mm}{0mm}{examplebordercolor}, arc=.5mm
}
\tcolorboxenvironment{claim}{
  enhanced jigsaw, pad at break*=1mm, breakable,
  left=4mm, right=4mm, top=1mm, bottom=1mm,
  colback=white, boxrule=0pt, frame hidden,
  borderline west={0.5mm}{0mm}{examplebordercolor}, arc=.5mm
}
\tcolorboxenvironment{fact}{
  enhanced jigsaw, pad at break*=1mm, breakable,
  left=4mm, right=4mm, top=1mm, bottom=1mm,
  colback=examplebackgroundcolor, boxrule=0pt, frame hidden,
  borderline west={0.5mm}{0mm}{examplebordercolor}, arc=.5mm
}

%%%%%%%%%%%%%%%%%%%%%%%%%
% Property Environments 					%
%%%%%%%%%%%%%%%%%%%%%%%%%

\theoremstyle{property}
\newtheorem{property}{Property}
\newtheorem{prop}{Proposition}
\newtheorem{result}{Result}

\tcolorboxenvironment{property}{
  enhanced jigsaw, pad at break*=1mm, breakable,
  left=4mm, right=4mm, top=1mm, bottom=1mm,
  colback=propertybackgroundcolor, boxrule=0pt, frame hidden,
  borderline west={0.5mm}{0mm}{propertybordercolor}, arc=.5mm
}
\tcolorboxenvironment{prop}{
  enhanced jigsaw, pad at break*=1mm, breakable,
  left=4mm, right=4mm, top=1mm, bottom=1mm,
  colback=propertybackgroundcolor, boxrule=0pt, frame hidden,
  borderline west={0.5mm}{0mm}{propertybordercolor}, arc=.5mm
}
\tcolorboxenvironment{result}{
  enhanced jigsaw, pad at break*=1mm, breakable,
  left=4mm, right=4mm, top=1mm, bottom=1mm,
  colback=propertybackgroundcolor, boxrule=0pt, frame hidden,
  borderline west={0.5mm}{0mm}{propertybordercolor}, arc=.5mm
}


%%%%%%%%%%%%
% Formulas 		%
%%%%%%%%%%%%

\theoremstyle{formula}
\newtheorem{formula}{Formula}

\tcolorboxenvironment{formula}{
  enhanced jigsaw, pad at break*=1mm, breakable,
  left=4mm, right=4mm, top=1mm, bottom=1mm,
  colback=formulabackgroundcolor, boxrule=0pt, frame hidden,
  borderline west={0.5mm}{0mm}{formulabordercolor}, arc=.5mm
}

%%%%%%%%%
% Proofs 		%
%%%%%%%%%

% These patches must be placed after \tcolorboxenvironment !
%% Change color of proof to match prev enviorment.
\AddToHook{env/theorem/after}{\colorlet{proofcolor}{theorembordercolor}}
\AddToHook{env/postulate/after}{\colorlet{proofcolor}{theorembordercolor}}
\AddToHook{env/conjecture/after}{\colorlet{proofcolor}{theorembordercolor}}
\AddToHook{env/corollary/after}{\colorlet{proofcolor}{theorembordercolor}}
\AddToHook{env/lemma/after}{\colorlet{proofcolor}{theorembordercolor}}
\AddToHook{env/conclusion/after}{\colorlet{proofcolor}{theorembordercolor}}

\AddToHook{env/definition/after}{\colorlet{proofcolor}{definitionbordercolor}}
\AddToHook{env/review/after}{\colorlet{proofcolor}{definitionbordercolor}}

\AddToHook{env/example/after}{\colorlet{proofcolor}{examplebordercolor}}
\AddToHook{env/remark/after}{\colorlet{proofcolor}{examplebordercolor}}
\AddToHook{env/note/after}{\colorlet{proofcolor}{examplebordercolor}}
\AddToHook{env/claim/after}{\colorlet{proofcolor}{examplebordercolor}}
\AddToHook{env/fact/after}{\colorlet{proofcolor}{examplebordercolor}}

\AddToHook{env/property/after}{\colorlet{proofcolor}{propertybordercolor}}
\AddToHook{env/prop/after}{\colorlet{proofcolor}{propertybordercolor}}
\AddToHook{env/result/after}{\colorlet{proofcolor}{propertybordercolor}}

\AddToHook{env/formula/after}{\colorlet{proofcolor}{formulabordercolor}}

\renewcommand{\qedsymbol}{Q.E.D.}
\let\qedsymbolMyOriginal\qedsymbol
\renewcommand{\qedsymbol}{
  \color{proofcolor}\qedsymbolMyOriginal
}

\newtheoremstyle{proof}
{0pt}{0pt}{\normalfont}{0pt}
{}{\;}{0.25em}
{{\sffamily\bfseries\color{proofcolor}\thmname{#1}.}
  \thmnote{\normalfont\color{black}~(\textit{#3})}}

\theoremstyle{proof}
\renewtheorem{proof}{Proof}

\tcolorboxenvironment{proof}{
  enhanced jigsaw, pad at break*=1mm, breakable,
  left=4mm, right=4mm, top=1mm, bottom=1mm,
  colback=white, boxrule=0pt, frame hidden,
  borderline west={0.5mm}{0mm}{proofcolor}, arc=.5mm
}

\newenvironment{info}{\begin{tcolorbox}[
      arc=0mm,
      colback=white,
      colframe=gray,
      title=Info,
      fonttitle=\sffamily,
      breakable
    ]}{\end{tcolorbox}}
\newenvironment{terminology}{\begin{tcolorbox}[
      arc=0mm,
      colback=white,
      colframe=green!60!black,
      title=Terminology,
      fonttitle=\sffamily,
      breakable
    ]}{\end{tcolorbox}}
\newenvironment{warning}{\begin{tcolorbox}[
      arc=0mm,
      colback=white,
      colframe=red,
      title=Warning,
      fonttitle=\sffamily,
      breakable
    ]}{\end{tcolorbox}}
\newenvironment{caution}{\begin{tcolorbox}[
      arc=0mm,
      colback=white,
      colframe=yellow,
      title=Caution,
      fonttitle=\sffamily,
      breakable
    ]}{\end{tcolorbox}}


%%%%%%%%%%%%%%%%%%%%
% Referencing setup				 %
%%%%%%%%%%%%%%%%%%%%
% setup hyperref
\usepackage[hidelinks]{hyperref}
\hypersetup{ linkcolor=black, filecolor=magenta, urlcolor=cyan, pdftitle={}, pdfpagemode=FullScreen, plainpages=false }

% Make numbering follow the chapter, and enviorment.
\numberwithin{equation}{chapter}
\numberwithin{figure}{chapter}
\numberwithin{table}{chapter}

\numberwithin{theorem}{chapter}
\numberwithin{postulate}{chapter}
\numberwithin{conjecture}{chapter}
\numberwithin{corollary}{chapter}
\numberwithin{lemma}{chapter}
\numberwithin{conclusion}{chapter}
\numberwithin{definition}{chapter}
\numberwithin{review}{chapter}
\numberwithin{example}{chapter}
\numberwithin{note}{chapter}
\numberwithin{claim}{chapter}
\numberwithin{fact}{chapter}
\numberwithin{property}{chapter}
\numberwithin{prop}{chapter}
\numberwithin{result}{chapter}
\numberwithin{formula}{chapter}

% Setup cleverref to custom enviorments
\usepackage[noabbrev, nameinlink,]{cleveref}
\crefname{postulate}{Postulate}{Postulates}
\crefname{conjecture}{Conjecture}{Conjectures}
\crefname{corollary}{Corollary}{Corollaries}
\crefname{lemma}{Lemma}{Lemmas}
\crefname{conclusion}{Conclusion}{Conclusions}
\crefname{definition}{Definition}{Definitions}
\crefname{review}{Review}{Reviews}
\crefname{example}{Example}{Examples}
\crefname{note}{Note}{Notes}
\crefname{claim}{Claim}{Claims}
\crefname{fact}{Fact}{Facts}
\crefname{property}{Property}{Properties}
\crefname{prop}{Proposition}{Propositions}
\crefname{formula}{Formula}{Formulas}


%%%%%%%%%%%%%%%%%%%%
% Header & Footer setup %
%%%%%%%%%%%%%%%%%%%%
\usepackage{fancyhdr}
\pagestyle{fancy}

% LE: left even
% RO: right odd
% CE, CO: center even, center odd
% My name for when I print my lecture notes to use for an open book exam.
% \fancyhead[LE,RO]{Gilles Castel}

\fancyhead[R]{\thetitle} % Right odd,  Left even
\fancyhead[L]{\theauthor}          % Right even, Left odd

\fancyfoot[R]{Page \thepage{} of {}\thelastpage}  % Right odd,  Left even
\fancyfoot[L]{\leftmark}          % Right even, Left odd
\fancyfoot[C]{}     % Center

\makeatother

%%%%%%%%%%%%%%%%%%%%
% Custom commands	 %
%%%%%%%%%%%%%%%%%%%%
\newcommand{\R}{\mathbb{R}}
\newcommand{\Q}{\mathbb{Q}}
\newcommand{\F}{\mathbb{F}}
\newcommand{\C}{\mathbb{C}}
\newcommand{\N}{\mathbb{N}}
\newcommand{\Z}{\mathbb{Z}}
\newcommand{\E}{\mathbb{E}}
\newcommand{\V}{\mathbb{V}}

\renewcommand{\P}{\mathbb{P}}
\newcommand\nulvec{\bm{0}}
\newcommand\nulmat{\bm{O}}
\newcommand{\vv}{{\bm{v}}}
\newcommand{\rr}{{\bm{r}}}
\newcommand{\uu}{\bm{u}}
\newcommand{\cc}{\bm{c}}
\newcommand{\ww}{\bm{w}}
\newcommand{\bfa}{\bm{a}}
\newcommand{\bfb}{\bm{b}}
\newcommand{\bfx}{\bm{x}}
\newcommand{\bb}[1]{\mathbb{#1}}
\DeclareMathOperator{\mat}{Mat}
\DeclareMathOperator{\D}{det}
\DeclareMathOperator{\matr}{M}
\DeclareMathOperator{\Mat}{Mat}
\DeclareMathOperator{\M}{Mat}
\DeclareMathOperator{\spn}{Span}
\DeclareMathOperator{\di}{dim}
\newcommand{\1}{\mathds{1}}

% Put x \to \infty below \lim
\let\svlim\lim\def\lim{\svlim\limits}

%Make implies and impliedby shorter
\let\implies\Rightarrow
\let\impliedby\Leftarrow
\let\iff\Leftrightarrow
\let\epsilon\varepsilon

% Add \contra symbol to denote contradiction
\usepackage{stmaryrd} % for \lightning
\newcommand\contra{\scalebox{1.5}{$\lightning$}}

%%%%%%%%%%%%%%%%%%%%
% Code formatting	 %
%%%%%%%%%%%%%%%%%%%%

\usepackage{listings} %Gør det muligt at indsætte pæn code

% Default fixed font does not support bold face
\DeclareFixedFont{\ttb}{T1}{txtt}{bx}{n}{10} % for bold
\DeclareFixedFont{\ttm}{T1}{txtt}{m}{n}{10}  % for normal


\definecolor{darkred}{rgb}{0.6,0.0,0.0}
\definecolor{darkgreen}{rgb}{0,0.50,0}
\definecolor{lightblue}{rgb}{0.0,0.42,0.91}
\definecolor{orange}{rgb}{0.99,0.48,0.13}
\definecolor{grass}{rgb}{0.18,0.80,0.18}
\definecolor{pink}{rgb}{0.97,0.15,0.45}


% General Setting of listings
\lstset{
  aboveskip=1em,
  breaklines=true,
  abovecaptionskip=-6pt,
  captionpos=b,
  escapeinside={\%*}{*)},
  frame=single,
  numbers=left,
  numbersep=15pt,
  numberstyle=\tiny,
}
% 0. Basic Color Theme
\lstdefinestyle{colored}{ %
  basicstyle=\ttfamily,
  backgroundcolor=\color{white},
  commentstyle=\color{green}\itshape,
  keywordstyle=\color{blue}\bfseries\itshape,
  stringstyle=\color{red},
}
% 1. General Python Keywords List
\lstdefinelanguage{PythonPlus}[]{Python}{
  morekeywords=[1]{,as,assert,nonlocal,with,yield,self,True,False,None,}, % Python builtin
  morekeywords=[2]{,__init__,__add__,__mul__,__div__,__sub__,__call__,__getitem__,__setitem__,__eq__,__ne__,__nonzero__,__rmul__,__radd__,__repr__,__str__,__get__,__truediv__,__pow__,__name__,__future__,__all__,predict,add,catagorical,learn,update,plot_images,read_labels,read_images,plot_images,linear_load,linear_save,image_to_vector,mean_square_error,argmax,evaluate,create_batches,col_space,row_space,isinstance,fact_mult,transpose,mat_mult,pow,flatten,reshape,}, % magic methods
  morekeywords=[3]{,object,type,isinstance,copy,deepcopy,zip,enumerate,reversed,list,set,len,dict,tuple,range,xrange,append,execfile,real,imag,reduce,str,repr,}, % common functions
  morekeywords=[4]{,Exception,NameError,IndexError,SyntaxError,TypeError,ValueError,OverflowError,ZeroDivisionError,}, % errors
  morekeywords=[5]{,ode,fsolve,sqrt,exp,sin,cos,arctan,arctan2,arccos,pi,
  array,norm,solve,dot,arange,isscalar,max,sum,shape,find,any,all,abs,plot,linspace,legend,quad,polyval,polyfit,hstack,concatenate,vstack,column_stack,empty,zeros,ones,rand,vander,grid,pcolor,eig,eigs,eigvals,svd,qr,tan,det,logspace,roll,min,mean,cumsum,cumprod,diff,vectorize,lstsq,cla,eye,xlabel,ylabel,squeeze,Matrix,LinAlg}, % numpy / math
}
% 2. New Language based on Python
\lstdefinelanguage{PyBrIM}[]{PythonPlus}{
  emph={d,E,a,Fc28,Fy,Fu,D,des,supplier,Material,Rectangle,PyElmt},
}
% 3. Extended theme
\lstdefinestyle{colorEX}{
  basicstyle=\ttfamily,
  backgroundcolor=\color{white},
  commentstyle=\color{darkgreen}\slshape,
  keywordstyle=\color{blue}\bfseries\itshape,
  keywordstyle=[2]\color{blue}\bfseries,
  keywordstyle=[3]\color{grass},
  keywordstyle=[4]\color{red},
  keywordstyle=[5]\color{orange},
  stringstyle=\color{darkred},
  emphstyle=\color{pink}\underbar,
  frame=tb,                         % Any extra options here
  showstringspaces=false,
}


% Python style for highlighting
\newcommand\pythonstyle{\lstset{
style=colorEX,
language=PyBrIM,
}}


% Python environment
\lstnewenvironment{python}[1][]
{
\pythonstyle
\lstset{#1}
}
{}


% Python for external files
\newcommand\pythonexternal[2][]{{
\pythonstyle
\lstinputlisting[#1]{#2}}}

% Python for inline
\newcommand\pythoninline[1]{{\pythonstyle\lstinline!#1!}}






\title{Final Project -- IPSA 2024}
\author{Emil B. A. \& Mathias K. N.}


\begin{document}

%%%%%%%%%%%%%%%%%%%%%%%%%%%%%%%%%%%%%%%%%%%%%%%%%%%%%%%%%%%%%%%%%%%%%%
%% Front page
%%%%%%%%%%%%%%%%%%%%%%%%%%%%%%%%%%%%%%%%%%%%%%%%%%%%%%%%%%%%%%%%%%%%%%

\thispagestyle{empty}
\setcounter{page}{0}

\begin{center}
  \huge
  \textbf{Introduction to Programming \\
  with Scientific Applications \\
  (Spring 2024)} \\[2ex]
  Final project \\[2cm]
\end{center}
\noindent
\begin{tabularx}{\textwidth}{|c|X|c|}
\multicolumn{1}{c}{Study ID} & 
\multicolumn{1}{l}{Name} & 
\multicolumn{1}{l}{\% contributed} \\
\hline
202204939 &Emil Beck Aagaard Korneliussen & 60 \\ % Student 1
\hline
202208528 &Mathias Kristoffer Nejsum  &40 \\ % Student 2
\hline
\end{tabularx}
\centerline{max 3 students}
\\[5ex]
\noindent
Briefly state the contributions of each of the group members to the project
\\
\begin{tabularx}{\textwidth}{|X|}
\hline
\bigskip
% Your comment goes here
Since Mathias had some handins due, before he was able to contribute to the
project, Emil started doing the first part. Therefore Emils contribution is
a bit higher than Mathias. 

Most of the work we did besides each other but, some code was written purely by Emil or
Mathias. 
\smallskip

Emil has written all the code which works with loading and saving of files,
as well as functions such as \pythoninline{predict, catagorical} and
\pythoninline{learn}. 
\smallskip

Mathias has written
all the code for the plotting the network, as well as the functions
\pythoninline{update} and \pythoninline{plot\_images} and also some matrix algebra. 
\bigskip
\\[3cm]
\hline 
\end{tabularx}


\vfill
\noindent
\textbf{Note on plagiarism} 
\\[2ex]
Since the evaluation of the project report and code will be part of the final grade in the course, \textbf{plagiarism in your project handin will be considered cheating at the exam}. Whenever adopting code or text from elsewhere you must state this and give a reference/link to your source. It is perfectly fine to search information and adopt partial solutions from the internet – actually, this is encouraged – but always state your source in your handin. Also discussing your problems with your project with other students is perfectly fine, but remember each group should handin their own solution. If you are in doubt if you solution will be very similar to another group because you discussed the details, please put a remark that you have discussed your solution with other groups.
\bigskip

\begin{raggedleft}
For more Aarhus University information on plagiarism, please visit\newline \url{http://library.au.dk/en/students/plagiarism/}
\end{raggedleft}
\newpage

%%%%%%%%%%%%%%%%%%%%%%%%%%%%%%%%%%%%%%%%%%%%%%%%%%%%%%%%%%%%%%%%%%%%%%

% Your report text goes here

\chapter{Introduction}  
\label{ch:introduction}

For our final project in Introduction to Programming with Scientific
Applications (IPSA), we have decided to do project IV on MNIST Image
Classification. In this project we will create a linear classifier that
identifies handwritten digits. We have written code for all mandatory questions in all three
parts 1-3.

% chapter (end)


\chapter{Discussion of code}  
\label{ch:structure_of_code}
This chapter is a serves as a introduction to the general codebase that we have
written. We will discuss both the design choices, dependencies and general
structure of the implementation, we will also discuss our main ideas for optimization.

\section{Structure of code}  
\label{sec:structure_of_code}

The MNIST project questions consists of three parts:
\begin{enumerate}
  \item Loading and saving of MNIST database files, and visualisation.
  \item Testing and evaluation of a set of weights for a linear classifier.
  \item Updating and learning a set of weights for a linear classifier.
\end{enumerate}
This provides a natural test based development approach to the project, since
code written in parts $1.$ and later $2.$ is used extensively to test any new
code written for the later parts. Naturally this progession is also used in the
structure of our codebase, reading from the top we first have imports such that
any dependencies are not hidden in the code base, then we have type hint
definitions which are used as abbreviations for specific types. These type hints
are used to make the code more readable, while providing a clear understanding
of both function argument types and return types. 

After these definitions the actual code begins, the functions appear in the same
order as they are described on the project page. Thus, as already mentioned any
function that can be used to test another function will be stated above that
function. As a specific example all the loading and saving of files is stated
before any function that utilizes the content of said files.

% section (end)
\section{Design choices}  
\label{sec:design_choices}

A major design choice of our codebase is that we have extracted all the linear
algebra functions into their own class contained in a separate file
\pythoninline{linalg.py}. This is a common practice during development of larger
codebases known as subprocess extraction, and it allows us to make the code more
readable and maintainable. The main goal was that we would define operations
such as matrix addition, scalar multiplication and matrix multiplication without
the need for appending a matrix object with \pythoninline{Matrix.add(Matrix)}.
To do this we have implemented a lot of dunder\footnote{abbreviation for double
underscore} methods. This allows us to write clear and concise functions, for
instance, have a look at the prediction function, in which a network consisting
of a weight matrix $A$, and a basis vector $b$ is used to generate a guess
vector:
\begin{python}
def predict(network: NetW, image: img) -> Matrix:
  x = image_to_vector(image)
  A = Matrix(network[0])
  b = Matrix(network[1])
  return x*A+b
\end{python}
By defining methods \pythoninline{__mul__} and \pythoninline{__add__} we can
effectively \emph{hide} list comprehensions in the well known operators * and
+. Thus, using this extraction principal, it becomes strikingly clear what the
prediction function does, which helps with debugging. 

One important design choice that we want to highlight in this linear algebra
module, is that we actually dont make a distinction between (row)vectors i.e.
1-dimensional lists and matrices, 2-dimensional lists. When we first started our
development, we actually did make that distinction, and therefore we initially
made two subclasses one for vectors and one for matrices. But when we started
actually using the module we discovered that the difference between the two
classes was miniscule. Honestly the fact that we had made a clear distinction
between the two types, lead to ugly code. A good example of this problem would
be when, we wanted to convert a row vector into a column vector, then we would
have to write:
\begin{python}
Matrix([Vector.elements]).transpose()
\end{python}
To solve this problem we wrote a new \pythoninline{Matrix.__init__()}
constructor to handle inputs of both 1- and 2-dimensional lists. One problem we
then had to fix was that the codebase has some code that can only be used on row
vectors, to accommodate any potential errors we decided to implement a boolean
property that all matrices have, appropriately named
\pythoninline{Matrix.row_vector}. Then any function that is only defined as
a row vector can just use and assertion statement to check that the provided
\pythoninline{Matrix} input is correct. This, new implementation did also fix
the before mentioned problem of converting row vectors to column vectors:
\begin{python}
row_vec = Matrix([x,...,z]) # Create a row vector using a 1D list 
col_vec = row_vec.transpose() 
\end{python}


% section (end)

\section{Dependencies}  
\label{sec:dependencies}
Random, matplotlib, gzip, json


% section (end)

\section{Visualisation}  
\label{ch:visualisation}



% section (end)


\section{Ideas for optimization}  
\label{sec:ideas_for_optimization}



% section (end)



% chapter (end)



\chapter{Reflection upon implementation}  
\label{ch:reflection_upon_implementation}

\section{Challenges during development}  
\label{sec:challenges_during_development}



% section (end)


% chapter (end)

\end{document}
