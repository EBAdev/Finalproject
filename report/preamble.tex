
%lav margins mindre (bedre til normale afleveringer)
\setlrmarginsandblock{2.5cm}{*}{1} \setulmarginsandblock{3cm}{3cm}{*} \checkandfixthelayout


\usepackage[utf8]{inputenc}  % Korrekt håndtering af æ, ø og å
\usepackage[T1]{fontenc}  % Korrekt håndtering af æ, ø og å
\usepackage{microtype}  % Typografi! Giver bl.a. pænere orddeling
\usepackage[english]{babel}  % Danske betegnelser og orddeling

\usepackage{amsmath, amssymb, bm, mathtools}  % Giver adgang til matematikting
\usepackage[Gray,squaren,thinqspace,thinspace]{SIunits} % Gør det muligt at angive enheder

\usepackage{graphicx}  % Gør det muligt at indsætte billeder
\usepackage{url} % Bruges til indsættelse af url direkte
\usepackage{float} % gør det muligt at tvinge en figurs placering med [H]
\usepackage{listings} %Gør det muligt at indsætte pæn code
\usepackage[small,bf,hang]{caption}  %Gør caption bold
\usepackage{subcaption} % Gør det muligt at lave delfigurer
\usepackage{lipsum}
\usepackage{dsfont} % giver adgang til\mathds{#} for at lave bb numre
\usepackage{parskip}

% TikZ
\usepackage{tikz} % Bruges til at lave flotte figurer
\usetikzlibrary{calc} % bruges til at lave nem placering i tikz
\usepackage{pgfplots} % bruges til plots i TikZ
\pgfplotsset{compat=1.18}

\tikzset{
    % Two node styles for game trees: solid and hollow
    solid node/.style={circle,draw,inner sep=1.5,fill=black},
    hollow node/.style={circle,draw,inner sep=1.5}
}

%%%%%%%%%%%%%%%%%%%%
% Begin enviorment setup			 %
%%%%%%%%%%%%%%%%%%%%
\usepackage{amsthm}
\usepackage{thmtools}
\usepackage[many]{tcolorbox}
\makeatletter

\def\renewtheorem#1{%
  \expandafter\let\csname#1\endcsname\relax
  \expandafter\let\csname c@#1\endcsname\relax
  \gdef\renewtheorem@envname{#1}
  \renewtheorem@secpar
}
\def\renewtheorem@secpar{\@ifnextchar[{\renewtheorem@numberedlike}{\renewtheorem@nonumberedlike}}
\def\renewtheorem@numberedlike[#1]#2{\newtheorem{\renewtheorem@envname}[#1]{#2}}
\def\renewtheorem@nonumberedlike#1{
  \def\renewtheorem@caption{#1}
  \edef\renewtheorem@nowithin{\noexpand\newtheorem{\renewtheorem@envname}{\renewtheorem@caption}}
  \renewtheorem@thirdpar
}
\def\renewtheorem@thirdpar{\@ifnextchar[{\renewtheorem@within}{\renewtheorem@nowithin}}
\def\renewtheorem@within[#1]{\renewtheorem@nowithin[#1]}

\makeatother

\usepackage[framemethod=TikZ]{mdframed}
\mdfsetup{skipabove=1em,skipbelow=0em, innertopmargin=12pt, innerbottommargin=8pt}

\tcbuselibrary{skins}

% Color definitions

\definecolor{proofcolor}{RGB}{0,0,0}

% Dark orange and Dark Red rgb
\definecolor{theorembordercolor}{RGB}{151, 63, 5}
\definecolor{theorembackgroundcolor}{RGB}{248, 241, 234}

\definecolor{examplebordercolor}{RGB}{0, 110, 184}
\definecolor{examplebackgroundcolor}{RGB}{240, 244, 250}

\definecolor{definitionbordercolor}{RGB}{0, 150, 85}
\definecolor{definitionbackgroundcolor}{RGB}{239, 247, 243}

\definecolor{propertybordercolor}{RGB}{128, 0, 128}
\definecolor{propertybackgroundcolor}{RGB}{255, 240, 255}

\definecolor{formulabordercolor}{RGB}{0, 0, 0}
\definecolor{formulabackgroundcolor}{RGB}{230, 229, 245}

%%% THEOREM STYLE SETUP %%%
\newtheoremstyle{theorem}
{0pt}{0pt}{\normalfont}{0pt}
{}{\;}{0.25em}
{{\sffamily\bfseries\color{theorembordercolor}\thmname{#1}~\thmnumber{\textup{#2}}.}
  \thmnote{\normalfont\color{black}~(#3)}}

\newtheoremstyle{definition}
{0pt}{0pt}{\normalfont}{0pt}
{}{\;}{0.25em}
{{\sffamily\bfseries\color{definitionbordercolor}\thmname{#1}~\thmnumber{\textup{#2}}.}
  \thmnote{\normalfont\color{black}~(#3)}}

\newtheoremstyle{example}
{0pt}{0pt}{\normalfont}{0pt}
{}{\;}{0.25em}
{{\sffamily\bfseries\color{examplebordercolor}\thmname{#1}~\thmnumber{\textup{#2}}.}
  \thmnote{\normalfont\color{black}~(#3)}}

\newtheoremstyle{property}
{0pt}{0pt}{\normalfont}{0pt}
{}{\;}{0.25em}
{{\sffamily\bfseries\color{propertybordercolor}\thmname{#1}~\thmnumber{\textup{#2}}.}
  \thmnote{\normalfont\color{black}~(#3)}}

\newtheoremstyle{formula}
{0pt}{0pt}{\normalfont}{0pt}
{}{\;}{0.25em}
{{\sffamily\bfseries\color{formulabordercolor}\thmname{#1}~\thmnumber{\textup{#2}}.}
  \thmnote{\normalfont\color{black}~(#3)}}

%%%%%%%%%%%%%%%%%%%%%%%%
% Theorem Environments 				%
%%%%%%%%%%%%%%%%%%%%%%%%

\theoremstyle{theorem}
\newtheorem{theorem}{Theorem}
\newtheorem{postulate}{Postulate}
\newtheorem{conjecture}{Conjecture}
\newtheorem{corollary}{Corollary}
\newtheorem{lemma}{Lemma}
\newtheorem{conclusion}{Conclusion}

\tcolorboxenvironment{theorem}{
  enhanced jigsaw, pad at break*=1mm, breakable,
  left=4mm, right=4mm, top=1mm, bottom=1mm,
  colback=theorembackgroundcolor, boxrule=0pt, frame hidden,
  borderline west={0.5mm}{0mm}{theorembordercolor}, arc=.5mm
}
\tcolorboxenvironment{postulate}{
  enhanced jigsaw, pad at break*=1mm, breakable,
  left=4mm, right=4mm, top=1mm, bottom=1mm,
  colback=theorembackgroundcolor, boxrule=0pt, frame hidden,
  borderline west={0.5mm}{0mm}{theorembordercolor}, arc=.5mm
}
\tcolorboxenvironment{conjecture}{
  enhanced jigsaw, pad at break*=1mm, breakable,
  left=4mm, right=4mm, top=1mm, bottom=1mm,
  colback=theorembackgroundcolor, boxrule=0pt, frame hidden,
  borderline west={0.5mm}{0mm}{theorembordercolor}, arc=.5mm
}
\tcolorboxenvironment{corollary}{
  enhanced jigsaw, pad at break*=1mm, breakable,
  left=4mm, right=4mm, top=1mm, bottom=1mm,
  colback=theorembackgroundcolor, boxrule=0pt, frame hidden,
  borderline west={0.5mm}{0mm}{theorembordercolor}, arc=.5mm
}
\tcolorboxenvironment{lemma}{
  enhanced jigsaw, pad at break*=1mm, breakable,
  left=4mm, right=4mm, top=1mm, bottom=1mm,
  colback=theorembackgroundcolor, boxrule=0pt, frame hidden,
  borderline west={0.5mm}{0mm}{theorembordercolor}, arc=.5mm
}
\tcolorboxenvironment{conclusion}{
  enhanced jigsaw, pad at break*=1mm, breakable,
  left=4mm, right=4mm, top=1mm, bottom=1mm,
  colback=theorembackgroundcolor, boxrule=0pt, frame hidden,
  borderline west={0.5mm}{0mm}{theorembordercolor}, arc=.5mm
}

%%%%%%%%%%%%%%%%%%%%%%%%%%%
% Definition Environments %
%%%%%%%%%%%%%%%%%%%%%%%%%%%

\theoremstyle{definition}
\newtheorem{definition}{Definition}
\newtheorem{review}{Review}

\tcolorboxenvironment{definition}{
  enhanced jigsaw, pad at break*=1mm, breakable,
  left=4mm, right=4mm, top=1mm, bottom=1mm,
  colback=definitionbackgroundcolor, boxrule=0pt, frame hidden,
  borderline west={0.5mm}{0mm}{definitionbordercolor}, arc=.5mm
}
\tcolorboxenvironment{review}{
  enhanced jigsaw, pad at break*=1mm, breakable,
  left=4mm, right=4mm, top=1mm, bottom=1mm,
  colback=definitionbackgroundcolor, boxrule=0pt, frame hidden,
  borderline west={0.5mm}{0mm}{definitionbordercolor}, arc=.5mm
}


%%%%%%%%%%%%%%%%%%%%%%%%
% Example Environments 			%
%%%%%%%%%%%%%%%%%%%%%%%%

\theoremstyle{example}
\newtheorem{example}{Example}
\newtheorem{remark}{Remark}
\newtheorem{note}{Note}
\newtheorem{claim}{Claim}
\newtheorem{fact}{Fact}

\tcolorboxenvironment{example}{
  enhanced jigsaw, pad at break*=1mm, breakable,
  left=4mm, right=4mm, top=1mm, bottom=1mm,
  colback=examplebackgroundcolor, boxrule=0pt, frame hidden,
  borderline west={0.5mm}{0mm}{examplebordercolor}, arc=.5mm
}
\tcolorboxenvironment{remark}{
  enhanced jigsaw, pad at break*=1mm, breakable,
  left=4mm, right=4mm, top=1mm, bottom=1mm,
  colback=white, boxrule=0pt, frame hidden,
  borderline west={0.5mm}{0mm}{examplebordercolor}, arc=.5mm
}
\tcolorboxenvironment{note}{
  enhanced jigsaw, pad at break*=1mm, breakable,
  left=4mm, right=4mm, top=1mm, bottom=1mm,
  colback=white, boxrule=0pt, frame hidden,
  borderline west={0.5mm}{0mm}{examplebordercolor}, arc=.5mm
}
\tcolorboxenvironment{claim}{
  enhanced jigsaw, pad at break*=1mm, breakable,
  left=4mm, right=4mm, top=1mm, bottom=1mm,
  colback=white, boxrule=0pt, frame hidden,
  borderline west={0.5mm}{0mm}{examplebordercolor}, arc=.5mm
}
\tcolorboxenvironment{fact}{
  enhanced jigsaw, pad at break*=1mm, breakable,
  left=4mm, right=4mm, top=1mm, bottom=1mm,
  colback=examplebackgroundcolor, boxrule=0pt, frame hidden,
  borderline west={0.5mm}{0mm}{examplebordercolor}, arc=.5mm
}

%%%%%%%%%%%%%%%%%%%%%%%%%
% Property Environments 					%
%%%%%%%%%%%%%%%%%%%%%%%%%

\theoremstyle{property}
\newtheorem{property}{Property}
\newtheorem{prop}{Proposition}
\newtheorem{result}{Result}

\tcolorboxenvironment{property}{
  enhanced jigsaw, pad at break*=1mm, breakable,
  left=4mm, right=4mm, top=1mm, bottom=1mm,
  colback=propertybackgroundcolor, boxrule=0pt, frame hidden,
  borderline west={0.5mm}{0mm}{propertybordercolor}, arc=.5mm
}
\tcolorboxenvironment{prop}{
  enhanced jigsaw, pad at break*=1mm, breakable,
  left=4mm, right=4mm, top=1mm, bottom=1mm,
  colback=propertybackgroundcolor, boxrule=0pt, frame hidden,
  borderline west={0.5mm}{0mm}{propertybordercolor}, arc=.5mm
}
\tcolorboxenvironment{result}{
  enhanced jigsaw, pad at break*=1mm, breakable,
  left=4mm, right=4mm, top=1mm, bottom=1mm,
  colback=propertybackgroundcolor, boxrule=0pt, frame hidden,
  borderline west={0.5mm}{0mm}{propertybordercolor}, arc=.5mm
}


%%%%%%%%%%%%
% Formulas 		%
%%%%%%%%%%%%

\theoremstyle{formula}
\newtheorem{formula}{Formula}

\tcolorboxenvironment{formula}{
  enhanced jigsaw, pad at break*=1mm, breakable,
  left=4mm, right=4mm, top=1mm, bottom=1mm,
  colback=formulabackgroundcolor, boxrule=0pt, frame hidden,
  borderline west={0.5mm}{0mm}{formulabordercolor}, arc=.5mm
}

%%%%%%%%%
% Proofs 		%
%%%%%%%%%

% These patches must be placed after \tcolorboxenvironment !
%% Change color of proof to match prev enviorment.
\AddToHook{env/theorem/after}{\colorlet{proofcolor}{theorembordercolor}}
\AddToHook{env/postulate/after}{\colorlet{proofcolor}{theorembordercolor}}
\AddToHook{env/conjecture/after}{\colorlet{proofcolor}{theorembordercolor}}
\AddToHook{env/corollary/after}{\colorlet{proofcolor}{theorembordercolor}}
\AddToHook{env/lemma/after}{\colorlet{proofcolor}{theorembordercolor}}
\AddToHook{env/conclusion/after}{\colorlet{proofcolor}{theorembordercolor}}

\AddToHook{env/definition/after}{\colorlet{proofcolor}{definitionbordercolor}}
\AddToHook{env/review/after}{\colorlet{proofcolor}{definitionbordercolor}}

\AddToHook{env/example/after}{\colorlet{proofcolor}{examplebordercolor}}
\AddToHook{env/remark/after}{\colorlet{proofcolor}{examplebordercolor}}
\AddToHook{env/note/after}{\colorlet{proofcolor}{examplebordercolor}}
\AddToHook{env/claim/after}{\colorlet{proofcolor}{examplebordercolor}}
\AddToHook{env/fact/after}{\colorlet{proofcolor}{examplebordercolor}}

\AddToHook{env/property/after}{\colorlet{proofcolor}{propertybordercolor}}
\AddToHook{env/prop/after}{\colorlet{proofcolor}{propertybordercolor}}
\AddToHook{env/result/after}{\colorlet{proofcolor}{propertybordercolor}}

\AddToHook{env/formula/after}{\colorlet{proofcolor}{formulabordercolor}}

\renewcommand{\qedsymbol}{Q.E.D.}
\let\qedsymbolMyOriginal\qedsymbol
\renewcommand{\qedsymbol}{
  \color{proofcolor}\qedsymbolMyOriginal
}

\newtheoremstyle{proof}
{0pt}{0pt}{\normalfont}{0pt}
{}{\;}{0.25em}
{{\sffamily\bfseries\color{proofcolor}\thmname{#1}.}
  \thmnote{\normalfont\color{black}~(\textit{#3})}}

\theoremstyle{proof}
\renewtheorem{proof}{Proof}

\tcolorboxenvironment{proof}{
  enhanced jigsaw, pad at break*=1mm, breakable,
  left=4mm, right=4mm, top=1mm, bottom=1mm,
  colback=white, boxrule=0pt, frame hidden,
  borderline west={0.5mm}{0mm}{proofcolor}, arc=.5mm
}

\newenvironment{info}{\begin{tcolorbox}[
      arc=0mm,
      colback=white,
      colframe=gray,
      title=Info,
      fonttitle=\sffamily,
      breakable
    ]}{\end{tcolorbox}}
\newenvironment{terminology}{\begin{tcolorbox}[
      arc=0mm,
      colback=white,
      colframe=green!60!black,
      title=Terminology,
      fonttitle=\sffamily,
      breakable
    ]}{\end{tcolorbox}}
\newenvironment{warning}{\begin{tcolorbox}[
      arc=0mm,
      colback=white,
      colframe=red,
      title=Warning,
      fonttitle=\sffamily,
      breakable
    ]}{\end{tcolorbox}}
\newenvironment{caution}{\begin{tcolorbox}[
      arc=0mm,
      colback=white,
      colframe=yellow,
      title=Caution,
      fonttitle=\sffamily,
      breakable
    ]}{\end{tcolorbox}}


%%%%%%%%%%%%%%%%%%%%
% Referencing setup				 %
%%%%%%%%%%%%%%%%%%%%
% setup hyperref
\usepackage[hidelinks]{hyperref}
\hypersetup{ linkcolor=black, filecolor=magenta, urlcolor=cyan, pdftitle={}, pdfpagemode=FullScreen, plainpages=false }

% Make numbering follow the chapter, and enviorment.
\numberwithin{equation}{chapter}
\numberwithin{figure}{chapter}
\numberwithin{table}{chapter}

\numberwithin{theorem}{chapter}
\numberwithin{postulate}{chapter}
\numberwithin{conjecture}{chapter}
\numberwithin{corollary}{chapter}
\numberwithin{lemma}{chapter}
\numberwithin{conclusion}{chapter}
\numberwithin{definition}{chapter}
\numberwithin{review}{chapter}
\numberwithin{example}{chapter}
\numberwithin{note}{chapter}
\numberwithin{claim}{chapter}
\numberwithin{fact}{chapter}
\numberwithin{property}{chapter}
\numberwithin{prop}{chapter}
\numberwithin{result}{chapter}
\numberwithin{formula}{chapter}

% Setup cleverref to custom enviorments
\usepackage[noabbrev, nameinlink,]{cleveref}
\crefname{postulate}{Postulate}{Postulates}
\crefname{conjecture}{Conjecture}{Conjectures}
\crefname{corollary}{Corollary}{Corollaries}
\crefname{lemma}{Lemma}{Lemmas}
\crefname{conclusion}{Conclusion}{Conclusions}
\crefname{definition}{Definition}{Definitions}
\crefname{review}{Review}{Reviews}
\crefname{example}{Example}{Examples}
\crefname{note}{Note}{Notes}
\crefname{claim}{Claim}{Claims}
\crefname{fact}{Fact}{Facts}
\crefname{property}{Property}{Properties}
\crefname{prop}{Proposition}{Propositions}
\crefname{formula}{Formula}{Formulas}


%%%%%%%%%%%%%%%%%%%%
% Header & Footer setup %
%%%%%%%%%%%%%%%%%%%%
\usepackage{fancyhdr}
\pagestyle{fancy}

% LE: left even
% RO: right odd
% CE, CO: center even, center odd
% My name for when I print my lecture notes to use for an open book exam.
% \fancyhead[LE,RO]{Gilles Castel}

\fancyhead[R]{\thetitle} % Right odd,  Left even
\fancyhead[L]{\theauthor}          % Right even, Left odd

\fancyfoot[R]{Page \thepage{} of {}\thelastpage}  % Right odd,  Left even
\fancyfoot[L]{\leftmark}          % Right even, Left odd
\fancyfoot[C]{}     % Center

\makeatother

%%%%%%%%%%%%%%%%%%%%
% Custom commands	 %
%%%%%%%%%%%%%%%%%%%%
\newcommand{\R}{\mathbb{R}}
\newcommand{\Q}{\mathbb{Q}}
\newcommand{\F}{\mathbb{F}}
\newcommand{\C}{\mathbb{C}}
\newcommand{\N}{\mathbb{N}}
\newcommand{\Z}{\mathbb{Z}}
\newcommand{\E}{\mathbb{E}}
\newcommand{\V}{\mathbb{V}}

\renewcommand{\P}{\mathbb{P}}
\newcommand\nulvec{\bm{0}}
\newcommand\nulmat{\bm{O}}
\newcommand{\vv}{{\bm{v}}}
\newcommand{\rr}{{\bm{r}}}
\newcommand{\uu}{\bm{u}}
\newcommand{\cc}{\bm{c}}
\newcommand{\ww}{\bm{w}}
\newcommand{\bfa}{\bm{a}}
\newcommand{\bfb}{\bm{b}}
\newcommand{\bfx}{\bm{x}}
\newcommand{\bb}[1]{\mathbb{#1}}
\DeclareMathOperator{\mat}{Mat}
\DeclareMathOperator{\D}{det}
\DeclareMathOperator{\matr}{M}
\DeclareMathOperator{\Mat}{Mat}
\DeclareMathOperator{\M}{Mat}
\DeclareMathOperator{\spn}{Span}
\DeclareMathOperator{\di}{dim}
\newcommand{\1}{\mathds{1}}

% Put x \to \infty below \lim
\let\svlim\lim\def\lim{\svlim\limits}

%Make implies and impliedby shorter
\let\implies\Rightarrow
\let\impliedby\Leftarrow
\let\iff\Leftrightarrow
\let\epsilon\varepsilon

% Add \contra symbol to denote contradiction
\usepackage{stmaryrd} % for \lightning
\newcommand\contra{\scalebox{1.5}{$\lightning$}}

